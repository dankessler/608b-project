\documentclass[12pt]{article}   % you have 10pt, 11pt, or 12pt options
\usepackage[margin=1in]{geometry}


\usepackage{physics}
\usepackage{amsmath}
\usepackage{amsmath,amsthm,amssymb,amsfonts,bbm,framed}
\usepackage{graphicx}
\usepackage{hyperref}
\usepackage{tikz}

\usetikzlibrary{bayesnet}
\usetikzlibrary{shapes,arrows}

\usepackage[url=false]{biblatex}
\bibliography{refs.bib}


%% custom macros
% vector
\newcommand{\V}[1]{\ensuremath{\boldsymbol{#1}}}
% matrix
\newcommand{\M}[1]{\ensuremath{#1}}
% math functions in non-italic font
\newcommand{\F}[1]{\ensuremath{\mathrm{#1}}}


\begin{document}  % necessary part of document


\title{Community Identification in Weighted Networks}
\author{Dan Kessler}

\maketitle


In this project I propose to investigate and (re)-implement procedures introduced in \cite{aicher_adapting_2013,aicher_learning_2015} for estimation of community labels for weighted stochastic block models, possibly extending these approaches to settings with a sample of networks and nodal covariates.

These papers consider a weighted extension of the stochastic block model (SBM) \cite{holland_stochastic_1983}, which is a frequently employed generative model used in the analysis of networks in a variety of settings.
In particular, the SBM provides a generative model for the construction of network data.
In the classic binary case with $n$ nodes, $c \in \left\{ 1, 2, \ldots, K \right\}^n$ is a random vector that serves to assign each of the nodes to one of $K$ ``communities.''
Conditional on $c$, the edges of the adjacency matrix $A_{i,j}$ are independent Bernoulli; all edges in a given ``block'' (i.e., all those edges connecting nodes from community $k$ to community $k'$) are further identically distributed with probability of connection given by $B_{k,k'}$.
In most settings, all that is observed is a single adjacency matrix $A$ and the task is to infer the community memberships $c$ and to subsequently estimate the entries of $B$.

The data that I consider in my research  is motivated by applications in cognitive neuroscience and is related to the networks generated by the stochastic block model, but with several important extensions.
First, I have been studying cases where we observe a \emph{sample} of networks on a common node set (or where the node set can be meaningfully registered across observations.
Second, the networks that I study are typically (i) dense, (ii) weighted, and (iii) signed.
Finally, I consider cases where in addition to observing the adjacency matrices $A^{(i)}$, we also observe nodal attributes $X^{(i)} \in \mathbb{R}^{n \times p}$, where $p$ is the number of covariates observed at each node.

While my current projects are largely focused on predicting observation-specific labels ($y_i)$ using $(A_i,X_i)$, in that work we currently consider $c$ fixed, known, and shared across all observations.
However, in future iterations of the project I intend to consider cases where $c$ is considered unknown or possibly cases where $c$ varies across the sample.
It has been suggested to me by my advisor that a better understanding of procedures that estimate $c$, which frequently involve some sort of stochastic optimization, in addition to developing an implementation myself, would be a useful tool for subsequent steps in my research, and that this class project is a good opportunity to undertake this work.

Unfortunately, direct estimation of $c$ is in most settings an $NP$-hard combinatorial problem, and so alternative strategies are employed.
In \cite{aicher_adapting_2013,aicher_learning_2015} the authors introduce variational Bayesian algorithms for estimating the posterior distribution of both the community labels and the parameters governing the edge distributions.
A reference implementation for these approaches is provided by the authors online, and as a pedagogical project I will re-implement their technique.
Depending on progress in this front, I will consider extensions that use Monte Carlo strategies from class in order to approximate for the posterior (rather than using variational methods).
If this proves successful, I will explore extensions of their model (and code) to the ``sample-of-networks'' settings described above.





\printbibliography

\end{document}



%%% Local Variables:
%%% mode: latex
%%% TeX-master: t
%%% End:

%  LocalWords:  variational
